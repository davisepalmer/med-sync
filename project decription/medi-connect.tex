\documentclass{article}

% Packages for styling
\usepackage[utf8]{inputenc}
\usepackage{geometry}
\usepackage{titlesec}
\usepackage{lipsum} % For dummy text, you can remove this in your actual document

% Customizing margins
\geometry{a4paper, margin=1in}

% Customizing section headings
\titleformat{\section}{\normalfont\Large\bfseries}{\thesection}{1em}{}
\titlespacing*{\section}{0pt}{\baselineskip}{\baselineskip}

\title{Medi-Connect}
\author{Leela Ravoori, Shasshank Sethuraman, Sahran Ashoor, and Davis Palmer}
\date{\today}

\begin{document}

\maketitle

\section*{Project Root}

In today's rapidly evolving healthcare landscape, efficient communication and collaboration among healthcare 
institutions have become \emph{significantly} important. With high-magnitude natural disasters and the threat
of wars shutting down healthcare systems of entire countries, and global emergencies of the likes of the
COVID-19 pandemic, we're now face-to-face with an admittantly daunting objective. Healthcare needs a 
'red alert' system. An actionable defense against the unexpected, accessible no matter where you are. In 
response, we'd like to present MediConnect, an application that facilitates a seamless sharing of 
resources, solicits second opinions on diagnoses, and brings out a unified platform of collaboration for
hospitals across the globe. By fostering interconnectedness among hospitals, MediConnect has the potential to
enables medical professionals to access critical resources swiftly in times of emergency. It additionally 
facilitates the exchange of expertise, ultimately leading to more accurate diagnoses and optimized treatment 
plans; moreover, in an era where medical knowledge proliferates at an unprecedented rate, having a centralized 
platform for sharing best practices and up-to-date medical literature becomes indispensable for healthcare 
practitioners striving to deliver the highest standard of care to their patients. Developing and 
utilizing such an application is an essential step towards revolutionizing healthcare delivery and improving 
patient outcomes globally.

\section*{Development Process}

Phase I saw our group begin a verbal brainstorm of ideas, accompanied by a combination of handwritten notes, 
projector demonstrations, and whiteboard flowcharts that outlined the rough framework and direction we expected to follow.
In anticipation of this direction, we prompty decided on the most appropriate tech stack, considering what goals
we had in mind. With the goal of a light-weight, data-oriented and easily-ammendable solution, we agreed to utlizie
a Python backend and HTML/CSS frontend under a Flask framework. With each team member being assigned to a specific
segment of our hurriedly-drawn 'To-Do' list, we capitalized on our strengths from prior experiences and began to code,
closely documenting our actions and communicating during vital source requests. As members had finished current 'To-Do' tasks,
they quickly picked up the next most pressing objective. Deliverables were prioritized, with additional features
implemented soon after.

\section*{Challenges and Obstacles}

As a team of students who have never met prior to the event, all of which hold very little experience performing
in hackathon settings, if at all, were tasked with quickly adapting to a new unfamiliar programming environment.
With a thirty-six hour clock soon to start ticking, our team chemistry had proven itself as we worked together
to familiarize ourselves with the Github environment. Through the development of our codebase, we faced frequent
roadblocks in honing common practices and etiquette when managing source requests, as unsychronized // unnecessary
push and pull requests easily shattered our mental timelines, abrubtly halting our flow of productivity. It took
time to familiarize ourselves with these environments, but as our communication developed, we established
a streamlined train of collaboration.

\section*{Acknowledgements}

\newpage
\section*{Works Cited}
\end{document}
